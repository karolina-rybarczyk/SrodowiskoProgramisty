\documentclass[12pt,a4paper]{article}

% ustawienia marginesu
\usepackage[left=1.6in,right=.8in,top=1.5in,bottom=1.5in]{geometry}

% polskie reguły dzielenia wyrazów itd
\usepackage{polski}

% polskie znaki zakodowane w UTF8
\usepackage[utf8]{inputenc}

% automatyczne generowanie odnośników w plikach PDF
\usepackage[pdftex,linkbordercolor={0 0.9 1}]{hyperref}
%
% pakiety matematyczne
\usepackage{amsthm,amsmath,amsfonts,amssymb,mathrsfs}

% ładne składanie odnośników do stron www
\usepackage{url}

% rozbudowane możliwości wypunktowań
\usepackage{enumerate}

% możliwość dodawania plików graficznych
\usepackage{graphicx}

%%% definicje twierdzeń itd :)
\newtheorem{tw}{Twierdzenie}[section]
\newtheorem{stw}[tw]{Stwierdzenie}
\newtheorem{fakt}[tw]{Fakt}
\newtheorem{lemat}[tw]{Lemat}

\theoremstyle{definition}
\newtheorem{df}[tw]{Definicja}
\newtheorem{ex}[tw]{Przykład}
\newtheorem{uw}[tw]{Uwaga}
\newtheorem{wn}[tw]{Wniosek}
\newtheorem{zad}{Zadanie}

% oznaczenia zbiorów liczbowych
\DeclareMathOperator{\R}{\mathbb{R}}
\DeclareMathOperator{\Z}{\mathbb{Z}}
\DeclareMathOperator{\N}{\mathbb{N}}
\DeclareMathOperator{\Q}{\mathbb{Q}}

% wartość bezwzględna, norma, iloczyn skalarny, nośnik, rozpięcie przestrzeni...
\providecommand{\abs}[1]{\left\lvert#1\right\rvert}
\providecommand{\var}[1]{\operatorname{var}(#1)}

% fajne nagłówki i stopki na stronie
\usepackage{fancyhdr}
\pagestyle{fancy}
\fancyhf{}
\fancyfoot[R]{\textbf{\thepage}}
\fancyhead[L]{\small\sffamily \nouppercase{\leftmark}}
\renewcommand{\headrulewidth}{0.4pt}
\renewcommand{\footrulewidth}{0.4pt}

% typowe dane dokumentu
\title{Funkcje ciągłe i różniczkowalne}
\date{14 Grudnia 2010}

% tu podaj swoje imię i nazwisko!
\author{Karolina Rybarczyk}

% zaczynamy dokument
\begin{document}
 
% pokaż tytuł
\maketitle

% spis treści
\tableofcontents

\section{Funkcje ciągłe}

\begin{df}
(funkcja ciągła). Niech $f:(a,b)\to\R$, oraz niech $x_0\in(a,b)$. Mówimy, że funkcja
$f$ jest ciągła w punkcie $x_0$ wtedy i tylko wtedy, gdy:
\[\forall_{\varepsilon<0}\exists_{\delta<0}\forall
x\in(a,b)|x-x_0|<\delta\Rightarrow|f(x)-f(x_0)|<\varepsilon\]
\end{df}

\begin{ex}
Wielomiany, funkcje trygonometryczne, wykładnicze, logarytmiczne są ciągłe w każdym
punkcie swojej dziedziny.
\end{ex}

\begin{ex}
Funkcja $f$ dana wzorem:
\[f(x)=\left\{\begin{array}{ll}x+1&\mbox{dla}\ x\neq0\\0&\mbox{dla}\
x=0\end{array}\right.\]
Jest ciągła w każdym punkcie poza $x_0 = 0$.
\\\\ Niech $\Q$ oznacza zbiór wszystkich liczb wymiernych.
\end{ex}

\begin{ex}
Funkcja $f$ dana wzorem:
\[f(x)=\left\{\begin{array}{ll}0&\mbox{dla}\ x\in\Q\\1&\mbox{dla}\
x\not\in\Q\end{array}\right.\]
nie jest ciągła w żadnym punkcie.
\end{ex}

\begin{ex}
Funkcja $f$ dana wzorem:
\[f(x)=\left\{\begin{array}{ll}0&\mbox{dla}\ x\in\Q\\x&\mbox{dla}\
x\not\in\Q\end{array}\right.\]
jest ciągła w punkcie $x_0=0$, ale nie jest ciągła w pozostałych punktach dziedziny.
\end{ex}

\begin{zad}
Udowodnij prawdziwość podanych przykładów.
\end{zad}

\begin{df}
Jeśli funkcja $f:A\to\R$ jest ciągła w każdym punkcie swojej dziedziny $A$ to mówimy
krótko, ze jest ciągła.
\end{df}

Poniższe twierdzenie zbiera podstawowe własności zbioru funkcji ciągłych.

\begin{tw}
Niech funkcje $f,g:R\to\R$ będą ciągłe, oraz niech $\alpha,\beta\in\R$. Wtedy
funkcje:
\begin{enumerate}[a)]
\item $h_1(x)=\alpha\cdot f(x)+\beta\cdot g(x)$,
\item $h_2(x)=f(x)\cdot g(x)$,
\item $h_3(x)=\frac{f(x)}{g(x)}$ (o ile $g(x)\not=0$ dla dowolnego $x\in\R$),
\item $h_4(x)=f(g(x))$,
\end{enumerate}
są ciągłe.
\end{tw}

Następne twierdzenie zwane powszechnie „własnością Darboux” lub twierdzeniem
o wartości pośredniej ma liczne praktyczne zastosowania. Mówi ono o tym,
ze jeśli funkcja ciągła przyjmuje jakieś dwie wartości, to przy odpowiednich
założeniach
co do dziedziny, przyjmuje tez wszystkie wartości pośrednie. Możemy sobie to
łatwo wyobrazić na przykładzie funkcji, która opisuje zmianę temperatury w czasie.
Jesli o 7:00 było $-1^{\circ}$C a o 9:00 było $2^{\circ}$C, to zapewne gdzieś między
7:00 a 9:00 był
taki moment, ze temperatura wynosiła dokładnie $0^{\circ}$C.

\begin{tw}
Niech $f:[a,b]\to\R$ ciągła, oraz niech $f(a)\not=f(b)$. Wtedy dla dowolnego $y_0\in
conv\{f(a),f(b)\}$ istnieje $x_0\in[a, b]$ takie, że $f(x_0)=y_0$.
\end{tw}

\section{Różniczkowalność}

\begin{df}
Niech $f:(a,b)\to\R$, $x_0\in(a,b)$ oraz $f$ ciągła w otoczeniu punktu $x_0$. Jeśli
istnieje granica:
\[\lim_{x\to x_0}\frac{f(x)-f(x_0)}{x-x_0}\]
i jest skończona, to oznaczamy ją przez $f'(x_0)$ i nazywamy pochodną funkcji $f$ w
punkcie $x_0$.
\end{df}

\begin{df}
Jeśli funkcja $f$ posiada pochodna w każdym punkcie swojej dziedziny, to mówimy, że
$f$ jest różniczkowalna. Istnieje wtedy funkcja $f'$, która każdemu punktowi z
dziedziny funkcji $f$ przyporządkowuje wartość pochodnej pochodnej funkcji $f$ w tym
punkcie.
\end{df}

\begin{ex}
Wielomiany, funkcje trygonometryczne, wykładnicze, logarytmiczne są różniczkowalne w
każdym punkcie dziedziny.
\end{ex}

\begin{ex}
Funkcja $f(x)=|x|$ jest ciągła, ale nie posiada pochodnej w punkcie $x_0=0$.
\end{ex}

\begin{tw}
Niech $f:[a,b]\to\R$ ciągła i różniczkowalna na $(a,b)$. Dodatkowo niech
$f'(x)\not=0$ dla $x\in(a, b)$, oraz niech $m=min_{x\in[a,b]}f(x)$, $M =
max_{x\in[a,b]}f(x)$. Wtedy na pewno $f(a)=m$, $f(b)=M$ lub $f(a)=M$ i $f(b)=m$.
\end{tw}

\end{document}
